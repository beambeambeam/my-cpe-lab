\documentclass{article}
\usepackage[margin=1in]{geometry}
\usepackage{array}
\usepackage{listings}
\usepackage{xcolor}

\lstset{
    basicstyle=\ttfamily\small,
    keywordstyle=\color{blue},
    commentstyle=\color{green!60!black},
    stringstyle=\color{red},
    numbers=left,
    numberstyle=\tiny,
    frame=single,
    breaklines=true
}

\begin{document}

\begin{center}
\textbf{Group P’Tae Space Force} \\
\end{center}
\begin{center}
  Jetanin Naitho 67070501011 \\
  Nantakorn Pinsupapron 67070501028 \\
  Supawit Marayat 67070501045
\end{center}

\section*{Problem 1}

\begin{center}
\renewcommand{\arraystretch}{1.8}
\begin{tabular}{|>{\centering\arraybackslash}p{2.5cm}|>{\centering\arraybackslash}p{2cm}|>{\centering\arraybackslash}p{2cm}|>{\centering\arraybackslash}p{2cm}|>{\centering\arraybackslash}p{2cm}|>{\centering\arraybackslash}p{2cm}|}
\hline
Ri/LOC & R1 & R2 & R3 & {[}100{]} & {[}104{]} \\
\hline
Initial & 5 & 10 & 0 & 20 & x \\
\hline
Inst 1. & 5 & 10 & 20 & 20 & x \\
\hline
Inst 2. & 30 & 10 & 20 & 20 & x \\
\hline
Inst 3. & 30 & 10 & 20 & 20 & x \\
\hline
Inst 4. & 30 & 10 & 20 & 20 & 10 \\
\hline
\end{tabular}
\end{center}

\section*{Problem 2}

\subsection*{C Language}

This is a C program that uses a `for` loop to compute the 44th Fibonacci number. The two previous numbers in the sequence are stored in variables `a` and `b`, and on each iteration, we compute the next Fibonacci number, update the variables, and continue until reaching the 44th term. The result is printed at the end.

\begin{lstlisting}[language=C]
#include <stdio.h>

int main() {
    long long a = 0, b = 1, temp;

    for (int i = 2; i <= 44; i++) {
        temp = a + b;
        a = b;
        b = temp;
    }

    printf("44th Fibonacci number: %lld\n", b);
    return 0;
}
\end{lstlisting}

\newpage

\subsection*{ARM Assembly}

This ARM Assembly code implements a loop to calculate the 44th Fibonacci number, similar to the C program shown above.

\begin{itemize}
    \item The registers R0 and R1 are used to store the previous two Fibonacci numbers (analogous to \texttt{a} and \texttt{b} in C).
    \item R2 is used as a temporary register to hold the sum, and R3 functions as the loop counter \texttt{i}.
    \item The loop iterates from \texttt{i = 2} up to 44, adding the previous two numbers, updating registers accordingly each time.
    \item After reaching the 44th iteration, the program prints the Fibonacci number using the \texttt{printf} function, referencing a string in the \texttt{.data} section.
\end{itemize}



\begin{lstlisting}
.global main
.text

main:
    PUSH {LR}
    MOV R0, #0          @ a = 0
    MOV R1, #1          @ b = 1
    MOV R3, #2          @ i = 2

loop:
    CMP R3, #44
    BGT done
    ADD R2, R0, R1      @ temp = a + b
    MOV R0, R1          @ a = b
    MOV R1, R2          @ b = temp
    ADD R3, R3, #1      @ i++
    B loop

done:
    LDR R0, =format
    BL printf
    MOV R0, #0
    POP {PC}

.data
format: .asciz "44th Fibonacci number: %d\n"
\end{lstlisting}

\subsection*{Converting C to ARM Assembly}

Command(s)/steps used to convert C into ARM assembly.

This assembly code was written (converted) by hand based on C source code. Already check and it run!.

\end{document}
